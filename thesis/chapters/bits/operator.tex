Similar to \Cref{def:power-law-potential}, we can define the (single) power law operator $\mathcal{Q}^\beta$:

% TODO: explain why we are not solving \int\int but only the following
\begin{definition}{Power Law Operator $\mathcal{Q}^\beta$}{power-law-operator}
  The power law operator $\mathcal{Q}^\beta: \functionspace \mapsto \functionspace$ is given by
  $$\mathcal{Q}^\beta[\rho](\vec{x}) := \int \norm{\vec{x}-\vec{y}}^\beta \,\dd\rho(\vec{y}) = \int_{\supp(\rho)} \norm{\vec{x}-\vec{y}}^\beta \rho(\vec{y}) \,\dd\vec{y}\,.$$
\end{definition}
% Either the attractive or the repulsive operator can be sparse.
% Obtained using \Cref{thm:theorem216}.

The operator $\mathcal{Q}^\beta$ acting on an equilibrium measure $\rho(\vec{x})$ returns the energy $\tilde{E}(\vec{x}) = \mathcal{Q}^\beta[\rho](\vec{x})$ at a point $\vec{x} \in B_1(\vec{0})$ in our normalised domain.

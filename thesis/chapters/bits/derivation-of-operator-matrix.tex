We will start by only considering a single power-law operator (out of two in the case of an attractive-repulsive interaction potential $K_{\alpha, \beta}$).
Substituting our ansatz given in \Cref{eq:ansatz} into $\hat{Q}[\rho]^\beta$, we obtain
\begin{equation}
  \hat{Q}^{\beta}(x) = \sum_{k=0}^{N-1} \rho_{k} \int_{B_1(\vec{0})} \norm{\vec{x}-\vec{y}}^{\beta} \weight{y}\jacobi{y} \,\dd\vec{y}\,,
  \label{eq:theorem216-in-derivation}
\end{equation}
luckily containing the integral evaluated in \Cref{thm:theorem216}.

We are now interested in a numerical representation of the operator $\hat{Q}^\beta$ acting on the function $\rho \in \functionspace$, so an equivalent (linear) operator $Q^\beta: \R^N \mapsto \R^N$ acting on the coefficients $\rho_k \in \R,\, k = 0, ..., N-1$.
As every finite-dimensional linear operator must have a matrix representation, we are looking for a $Q^\beta \in \R^{N \times N}$ such that
$$\hat{Q}^\beta[\rho](\vec{x}) = \jacobivec{x} \cdot Q^\beta \vec{\rho}\,,$$
where $\jacobivec{x} \in \R^N$ is the vector of Jacobi polynomials $P^{(a, b)}_0(x)$, $P^{(a, b)}_1(x)$, ..., $P^{(a, b)}_{N-1}(x)$ evaluated at $2\norm{\vec{x}}^2 - 1$ as introduced in and after \Cref{def:jacobi-polynomials}.
Note that in the context of linear combinations of Jacobi polynomials, we will use zero-based indexing for vectors and matrices due to the convention that the first orthogonal polynomial is usually denoted by $p_0(x) = 1$, in line with $\deg(p_k) = k$.

Based on the Three-Term Recurrence Relationship (cf. \Cref{thm:three-term-recurrence-relationship}), one can even determine an explicit relationship between the coefficients in the Jacobi expansion by considering the Jacobi Matrix (cf. \Cref{remark:jacobi-matrix}).

% TODO: this section has weird indexing (start at 0 or 1?) and needs some work.
Therefore, starting from \Cref{eq:theorem216-in-derivation}, we obtain
\begin{align*}
  \hat{Q}^\beta[\rho](\vec{x}) & = \sum_{k=0}^{N-1} \rho_k \hat{Q}^\beta[wP_k](\vec{x}) = \sum_{k=0}^{N-1} \rho_k \sum_{j=0}^{N-1} q_{kj} \jacobi{x} \\
                               & = \sum_{j=0}^{N-1} \sum_{k=0}^{N-1} \rho_k q_{kj} \jacobi{x}\,,
\end{align*}
which we will rewrite in matrix-form,
\begin{align*}
  \hat{Q}^\beta[\rho](\vec{x}) & = \vec{P}(\vec{x}) \cdot
  \begin{pmatrix}
    \sum_{k=0}^{N-1} \rho_k q_{k,1} \\
    \vdots                          \\
    \sum_{k=0}^{N-1} \rho_k q_{k,N}
  \end{pmatrix} = \vec{P}(\vec{x}) \cdot
  \underbrace{
    \begin{pmatrix}
      q_{00}    & \dots  & q_{0,N-1}   \\
      \vdots    & \ddots & \vdots      \\
      q_{N-1,0} & \dots  & q_{N-1,N-1} \\
    \end{pmatrix}
  }_{=: Q^\beta}
  \begin{pmatrix}
    \rho_0 \\
    \vdots \\
    \rho_{N-1}
  \end{pmatrix}                                                          \\
                               & = \jacobivec{x} \cdot Q^\beta \vec{\rho}
\end{align*}
where we used $\vec{P}(\vec{x}) = \jacobivec{x}$ as a shorthand
giving us the form of the operator matrix.
Each value $q_{kj}$ in it is therefore chosen to satisfy ...
\hierKoennteIhreWerbungStehen

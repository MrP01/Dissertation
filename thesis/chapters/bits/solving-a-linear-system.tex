\section{Solving a Linear System}
Once the operator is computed, we are now looking for a set of solution coefficients $\vec{\rho} \in \R^N$ such that the total energy $E = E_{\rm kin} + U = U = U_K[\hat{\rho}]$ (in the presence of friction, the kinetic energy will eventually dissipate, cf. \Cref{chap:particle-interaction-theory}) on the domain $D = B_R(\vec{0})$ is constant.
That means, we are looking for $\vec{\rho} \in \R^N$ such that
\begin{equation}
  \mathcal{Q}[\rho](\vec{x}) = \tilde{E}(\vec{x}) = E\,,
\end{equation}
where we can expand $\tilde{E}(\vec{x}) = \jacobivec{x} \cdot \vec{E}$ into Jacobi polynomials with coefficients $\vec{E} = E \vec{e}_1 = (E, 0, ..., 0)^T$ such that the energy is constant along the entire domain, so $\tilde{E}(\vec{x}) = E \cdot P_0^{(a, b)}\left(\jacobiarg{x}\right) = E$.
In matrix form, that is
$$Q \vec{\rho} = \vec{E} \qLRq \begin{pmatrix}
    q_{00}    & \dots  & q_{0,N-1}   \\
    \vdots    & \ddots & \vdots      \\
    q_{N-1,0} & \dots  & q_{N-1,N-1} \\
  \end{pmatrix} \begin{pmatrix}
    \rho_0 \\
    \vdots \\
    \rho_{N-1}
  \end{pmatrix} = \begin{pmatrix}
    E       \\
    \vec{0} \\
    0
  \end{pmatrix}\,.$$

This equation $Q \vec{\rho} = \vec{E}$ contains two unknowns, so we need a second equation to find the full solution $\rho \in \functionspace$ and, thereby, $\hat{\rho} \in \functionspacehat$.
The second piece of information we are looking for is the mass given in \Cref{eq:measure-mass}, which is set to $M = 1$. We start by dividing $Q \vec{\rho} = E \vec{e}_1$ by the unknown energy $E$,
$$Q \frac{\vec{\rho}}{E} = Q \vec{\tilde{\rho}} = \vec{e}_1\,,$$
which we can efficiently solve using readily available linear system solvers.
After solving, we ensure $M\left[\rho[\vec{\tilde{\rho}}]\right] = 1$, using \Cref{lemma:mass}, leading us to our final equilibrium distribution $\rho \in \functionspace$.

% The linear system is solved using a QR-decomposition.
% TODO: what would Yuji say?

Hence the total potential (energy) of a given solution $\vec{\rho}$ is obtained by
\begin{equation}
  E(R) = \{Q_{\alpha,\beta} \vec{\rho}\}_1 = \sum_{k=0}^{N-1} \rho_k \left(\frac{R^{\alpha+d}}{\alpha} \bar{I}_{m,k}^{\alpha,\alpha} - \frac{R^{\beta+d}}{\beta} \bar{I}_{m,k}^{\alpha,\beta}\right)\,,
  \label{eq:total-energy-for-ansatz}
\end{equation}
with $\bar{I}_{m,n}^{\alpha,\alpha}$ the 0th coefficient of a Jacobi polynomial expansion of $I_{m,n}^{\alpha,\alpha}(\vec{x})$ (cf. \Cref{thm:theorem216}) as given in \Cref{eq:0th-coeff-of-I}.

\subsection{Tikhonov Regularisation}
The normal system is: ... \\
Also referred to as Ridge regression
% TODO

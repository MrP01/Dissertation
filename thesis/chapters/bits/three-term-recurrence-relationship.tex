\begin{theorem}{Three-Term Recurrence Relationship}{three-term-recurrence-relationship}
  All orthogonal polynomials $\{p_0, p_1, p_2, ...\}$ (cf. \Cref{def:orthogonal-polynomials}) have (at least) a three-term recurrence relationship of the form
  $$A_n p_{n+1}(x) = (B_n - x) p_n(x) + C_n p_{n-1}(x)\,.$$
\end{theorem}
\begin{proof}
  For readability, let ``$x p_n$''$:= x \mapsto x p_n(x)$ denote the function resulting from multiplication of the polynomial $p_n$ by its variable, resulting in a polynomial of $\deg(x p_n) \le n+1$.
  By the linear independence of all orthogonal polynomials $p_n$ with respect to the inner product $\langle \cdot, \cdot \rangle_w$, it must be possible to write
  $$x p_n(x) = \sum_{k=0}^{n+1} \hat{a}_k p_k(x)\,,\quad \text{for some}~\hat{a}_k \in \R, k=0, ..., n+1\,.$$
  Now, for all $n \ge 0$ and $m \le n+1$ we have
  $$\langle xp_n, p_m \rangle_w = \sum_{k=0}^{n+1} \hat{a}_k \langle p_k, p_m \rangle_w = \sum_{k=0}^{n+1} \hat{a}_k \delta_{i,k} = \hat{a}_m \langle p_m, p_m \rangle_w\,,$$
  due to the orthogonality relationship (\Cref{thm:jacobi-orthogonality-condition}).
  Therefore,
  \begin{equation}
    \hat{a}_m = \frac{\langle xp_n, p_m \rangle_w}{\langle p_m, p_m \rangle_w} \quad\text{for all}~m \le n+1\,.
    \label{eq:three-term-step}
  \end{equation}

  However, when \underline{$m < n-1$}, we have $\deg(xp_m) < n$ so $x p_m(x) = \sum_{k=0}^{n-1} \hat{b}_k p_k(x)$ for some (potentially 0) $\hat{b}_k \in \R$, and therefore $\langle p_n, xp_m \rangle_w = \sum_{k=0}^{n-1} \hat{b}_k \langle p_n, p_k \rangle_w = 0$,
  which, by the symmetry of the inner product (\Cref{remark:symmetry-of-inner-product}), also implies $\langle x p_n, p_m \rangle_w = 0$ which, by \Cref{eq:three-term-step}, allows us to conclude that the earlier coefficients $\hat{a}_m = 0$.

  % And obviously, even if we assumed higher-order coefficients in the expansion of $xp_n$, when $m > n+1 \Leftrightarrow n < m-1$, $\deg(xp_n) < m$ and so all later coefficients $\hat{a}_m = 0$.

  We recall that $xp_n(x) = \sum_{k=0}^{n+1} \hat{a}_k p_k(x)$, which in combination with our insights on the $\hat{a}_m$ above means that
  $$xp_n(x) = \hat{a}_{n-1} p_{n-1}(x) + \hat{a}_{n} p_n(x) + \hat{a}_{n+1} p_{n+1}(x)\,,$$
  concluding the proof.
\end{proof}

For example, for the Chebyshev polynomials $T_k: [-1, 1] \to \R$ we have
$$T_{k+1}(x) = 2x T_k(x) - T_{k-1}(x) \,.$$

Note that the converse of \Cref{thm:three-term-recurrence-relationship} is also true, a set of polynomials of increasing degree $k$ that has a three-term recurrence relationship is a set of orthogonal polynomials (cf. \Cref{def:orthogonal-polynomials}).
While the original theorem is believed to be discovered by Thomas Joannes Steltjes, so before \cite{1935-favard}, to this day we still refer to it as \textit{Favard's theorem}.

% From HeatFun:
% \begin{theorem}{Chebyshev recursion formula}{chebrecursion}
%   The \chebyshev polyomials satisfy the three-term recurrence relation $$T_{k+1}(x) = 2x T_k(x) - T_{k-1}(x) \,.$$
% \end{theorem}
% \begin{proof}{\parencite{atap}.}
%   For $k > 1$, we have
%   \begin{align*}
%     2x T_k(x) - T_{k-1}(x) & = 2x \cdot \frac{1}{2} (z^k + z^{-k}) - \frac{1}{2} (z^{k-1} + z^{-(k-1)})                     \\
%                            & = 2 \frac{1}{2}(z + z^{-1}) \cdot \frac{1}{2}(z^k + z^{-k}) - \frac{1}{2} (z^{k-1} + z^{-k+1}) \\
%                            & = \frac{1}{2} (z^{k+1} + z^{k-1} + z^{-k+1} + z^{-k-1}) - \frac{1}{2} (z^{k-1} + z^{-k+1})     \\
%                            & = \frac{1}{2} (z^{k+1} + z^{-(k+1)}) = T_{k+1}(x)
%   \end{align*}
% \end{proof}

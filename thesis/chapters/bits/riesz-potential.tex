Let $(-\Delta)^{-\gamma}$ denote the inverse fractional Laplacian $\Delta := \nabla^2$ with power $\gamma \in (0, 1)$.
There are numerous equivalent definitions available (cf. \cite{2015-fractional-laplacian-definitions}), within the context of potential theory the Riesz potential definition (\Cref{def:riesz-potential}) is the most common.

\begin{definition}{Riesz Potential}{riesz-potential}
  For a given function $u: \R^d \to \R$ and $\gamma \in \R$, its \textit{Riesz potential} $I_{\gamma}[u]$ is given by
  $$I_{\gamma}[u](\vec{x}) := \frac{2^{-\gamma} \Gamma(\tfrac{d-\gamma}{2})}{\pi^{d/2} \Gamma(\gamma/2)} \int_{\R^d} \frac{u(\vec{z})}{\norm{\vec{x}-\vec{z}}^{d-\gamma}} \dd\vec{z}\,.$$
\end{definition}

For $\gamma \in (0, d)$, the Riesz potential is equivalent to the inverse fractional Laplacian, so $(-\Delta)^{-\gamma} = I_\gamma$.
So in the case of positive power law kernels, the equivalence to the inverse fractional Laplacian does not apply.
Results on the integral will hold nevertheless and we move on to stating \Cref{thm:theorem216} from \cite{2021-arbitrary-dimensions} verbatim.

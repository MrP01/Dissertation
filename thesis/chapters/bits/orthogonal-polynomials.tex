\begin{definition}{Orthogonal Polynomials}{orthogonal-polynomials}
  Orthogonal polynomials are univariate polynomials
  $p_n: \R \to \R, \; p_n(x) = \sum_{k=0}^{n} c_k x^k$, $n \in \N_0$,
  that form an orthogonal basis under the inner product $\langle p_n, p_m \rangle_w$ with weight function $w(x)$, given by
  $$\langle f, g \rangle_w := \int_{D_p} f(x) g(x) w(x) \,\ddx\,,$$
  the integral over some domain $D_p \subseteq \R$.
\end{definition}

The domain of the integral, for all intents and purposes within this dissertation, will be the Chebyshev interval $D_p = [-1, 1]$.

\begin{remark}{}{symmetry-of-inner-product}
  Under this inner product, multiplication by the variable $x$ is self-adjoint, so it satisfies $\langle x\mapsto xf(x), g \rangle_w = \langle f, x \mapsto xg(x)\rangle_w$.
\end{remark}

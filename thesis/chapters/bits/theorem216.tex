\begin{theorem}{Power Law Potential of the $n$th Jacobi Polynomial}{theorem216}
  On the $d$-dimensional unit ball $B_1$ the power law potential, with power $\alpha \in(-d,2+2m-d)$, $m\in\mathbb{N}_0$ and $\beta>-d$, of the $n$-th weighted radial Jacobi polynomial $(1-\norm{\vec{y}}^2)^{m-\frac{\alpha+d}{2}}P_n^{\left(m-\frac{\alpha+d}{2},\frac{d-2}{2}\right)}\left(\jacobiarg{y}\right)$ reduces to a Gaussian hypergeometric function as follows:
  \begin{align*}
    I_{m,n}^{\alpha,\beta}(\vec{x}) & = \int_{B_1(\vec{0})} \norm{\vec{x}-\vec{y}}^\beta \left(1-\norm{\vec{y}}^2\right)^{m-\frac{\alpha+d}{2}} P_{n}^{\left(m-\frac{\alpha+d}{2},\frac{d-2}{2}\right)}\left(\jacobiarg{y}\right) \dd\vec{y}                                                                                                                                                                                                             \\
                                    & = \tfrac{\pi ^{d/2} \Gamma \left(1+\frac{\beta}{2}\right) \Gamma \left(\frac{\beta+d}{2}\right) \Gamma \left(m+n-\frac{\alpha+d}{2}+1\right)}{\Gamma \left(\frac{d}{2}\right) \Gamma (n+1) \Gamma \left(\frac{\beta}{2}-n+1\right) \Gamma \left(\frac{\beta-\alpha}{2}+m+n+1\right)}{}_2F_1\left(\begin{matrix}n-\frac{\beta}{2}, -m-n+\frac{\alpha-\beta}{2} \\\frac{d}{2}\end{matrix};\norm{\vec{x}}^2\right)\,.
  \end{align*}
\end{theorem}

\begin{proof}[Proof (adapted from \cite{2021-arbitrary-dimensions})]
  We begin by applying \Cref{lemma:jacobi-polynomial-series} to the inside of the integrand.
  \begin{align*}
    I := \int_{B_1(\vec{0})} & \norm{\vec{x}-\vec{y}}^\beta \weight{y} \jacobi[n]{y} \dd\vec{y}                                                                                             \\
                             & = C_{a,b,n} \sum_{k=0}^n \binom{n}{k} C_{a,b,n,k} \int_{B_1(\vec{0})} \norm{\vec{x}-\vec{y}}^\beta \weight{y} \left(\norm{\vec{y}}^2 - 1\right)^k \dd\vec{y} \\
                             & = C_{a,b,n} \sum_{k=0}^n \binom{n}{k} C_{a,b,n,k} (-1)^k \int_{B_1(\vec{0})} \norm{\vec{x}-\vec{y}}^\beta \weight[a+k]{y} \dd\vec{y}
  \end{align*}
  where $a := m-\frac{\alpha+d}{2}$ and $b := \frac{d-2}{2}$.
  Note that from the first to the second line, we used $\frac{z-1}{2} = \frac{2\norm{\vec{y}}^2-1 - 1}{2} = \norm{\vec{y}}^2 - 1$.

  The constants are
  \begin{align*}
    C_{a,b,n}   & := \frac{\Gamma(a+1+n)}{n! \Gamma(1+a+b+n)}     \\
    C_{a,b,n,k} & := \frac{\Gamma(1+a+b+n + k)}{\Gamma(a+1+k)}\,.
  \end{align*}

  We identify the remaining integral as the Riesz potential $I_{\beta+d}[u](\vec{x})$, cf. \Cref{def:riesz-potential}, of the function $u(\vec{y}) := \left(1-\norm{\vec{y}}^2\right)^{a+k}$, which we can evaluate using Lemma 2.4 from \cite{2011-porous-medium-1}:
  \begin{align*}
    \int_{B_1(\vec{0})} & \norm{\vec{x}-\vec{y}}^\beta \weight[a+k]{y} \dd\vec{y} = c_{\beta+d} I_{\beta+d}\left[\vec{y} \mapsto \left(1-\norm{\vec{y}}^2\right)^{a+k}\right](\vec{x}) \\
                        & = c_{\beta+d} C_{a+k, \beta, d} \cdot {}_2F_1\left(\begin{matrix}\frac{d-(\beta+d)}{2}, -a-k-\frac{\beta+d}{2} \\d/2\end{matrix}; \norm{\vec{x}}^2\right)    \\
                        & = c_{\beta+d} C_{a+k, \beta, d} \cdot {}_2F_1\left(\begin{matrix}-\beta/2, -m-k+\frac{\alpha-\beta}{2} \\d/2\end{matrix}; \norm{\vec{x}}^2\right)\,,
  \end{align*}
  as $-a-k-\frac{\beta+d}{2} = -m + \frac{\alpha+d}{2} -k - \frac{\beta+d}{2} = -m-k+\frac{\alpha-\beta}{2}$ with constants
  \begin{align*}
    c_{\beta+d}       & := \frac{2^{\beta+d} \pi^{d/2} \Gamma\left(\frac{\beta+d}{2}\right)}{\Gamma(-\beta/2)}                      \quad \text{\textcolor{gray}{from aforementioned definition of the Riesz potential}} \\
    C_{a+k, \beta, d} & := \frac{\Gamma(a+k+1) \Gamma(-\beta/2)}{2^{\beta+d}\Gamma(d/2) \Gamma\left(a+k+\frac{\beta+d}{2}+1\right)} \quad \text{\textcolor{gray}{from Lemma 2.4}}\,,
  \end{align*}
  and therefore
  $$c_{\beta+d} C_{a+k, \beta, d} = \frac{\cancel{2^{\beta+d}} \pi^{d/2} \Gamma\left(\frac{\beta+d}{2}\right) \Gamma(a+k+1) \cancel{\Gamma(-\beta/2)}}{\cancel{\Gamma(-\beta/2)} \cancel{2^{\beta+d}}\Gamma(d/2) \Gamma\left(a+k+\frac{\beta+d}{2}+1\right)} = \frac{\pi^{d/2}B\left(\tfrac{\beta+d}{2}, a+k+1\right)}{\Gamma(d/2)}\,,$$
  using \Cref{eq:beta-gamma}.
  So that finally,
  \begin{align*}
    \int_{B_1(\vec{0})} & \norm{\vec{x}-\vec{y}}^\beta \weight[a+k]{y} \dd\vec{y}                                                                                                                                                         \\
                        & =\frac{\pi^{d/2}}{\Gamma(d/2)} B\left(\tfrac{\beta+d}{2}, m-\tfrac{\alpha+d}{2}+k+1\right) \cdot {}_2F_1\left(\begin{matrix}-\beta/2, -m-k+\frac{\alpha-\beta}{2} \\d/2\end{matrix}; \norm{\vec{x}}^2\right)\,.
  \end{align*}

  Plugging this back into the original form above, carrying along the same parameters,
  \begin{align*}
    I = C_{a,b,n} \sum_{k=0}^n \binom{n}{k} C_{a,b,n,k} (-1)^k \frac{\pi^{d/2}}{\Gamma(d/2)} B\left(\cdot, \cdot\right) {}_2F_1\left(\dots; \norm{\vec{x}}^2\right)\,
  \end{align*}
  we can apply Equation (2.1) in \cite{2021-arbitrary-dimensions} after some algebra, the special case of an identity given in \cite{1986-crazy-hypergeometric-properties} to obtain a ${}_3F_2$ (three terms in the numerator, two in the denominator) function
  \begin{align*}
    I \propto {}_3F_2\left(\begin{matrix}-\beta/2, n-\beta/2, -m-n+\frac{\alpha-\beta}{2} \\d/2, -\beta/2\end{matrix}; \norm{\vec{x}}^2\right)\,,
  \end{align*}
  which we expand into its \Cref{def:generalised-hypergeometric-series} to see that two terms cancel:
  \begin{align*}
    I \propto \sum_{k=0}^{\infty} \frac{\cancel{(-\beta/2)_k}, (n-\beta/2)_k, \left(-m-n+\frac{\alpha-\beta}{2}\right)_k}{(d/2)_k \cancel{(-\beta/2)_k}} \frac{\norm{\vec{x}}^{2k}}{k!}\,,
  \end{align*}
  which results back in a ${}_2F_1$ function (two terms in the numerator, one in the denominator), the so-called Gaussian hypergeometric function, cf. \Cref{lemma:gaussian-hypergeometric-function}, and after combining $C_{a,b,n}$, $C_{a,b,n,k}$, $\frac{\pi^{d/2}}{\Gamma(d/2)}$ with the gamma-function expansion of $B\left(\tfrac{\beta+d}{2}, m-\tfrac{\alpha+d}{2}+k+1\right)$ according to \Cref{eq:beta-gamma}, and cancelling terms, one finally obtains
  $$I = \tfrac{\pi^{d/2} \Gamma \left(1+\frac{\beta}{2}\right) \Gamma \left(\frac{\beta+d}{2}\right) \Gamma \left(m+n-\frac{\alpha+d}{2}+1\right)}{\Gamma \left(\frac{d}{2}\right) \Gamma (n+1) \Gamma \left(\frac{\beta}{2}-n+1\right) \Gamma \left(\frac{\beta-\alpha}{2}+m+n+1\right)}{}_2F_1\left(\begin{matrix}n-\frac{\beta}{2}, -m-n+\frac{\alpha-\beta}{2} \\\frac{d}{2}\end{matrix};\norm{\vec{x}}^2\right)\,,$$
  concluding the proof.
\end{proof}

Lemma 2.4 from \cite{2011-porous-medium-1,1967-formulas-and-theorems} is based on the \textit{Weber-Schafheitlin} integral of two Bessel functions given in \cite{1945-bessel-integral}.
The Weber-Schafheitlin integrals are related to the fractional Laplacians of aforementioned functions because the Fourier transform of ${}_2F_1$ is a Bessel function.
For a more generalised version of Lemma 2.4, see \cite{2014-barenblatt}.

Also note that for even integer $\beta$, the prefactor in \Cref{thm:theorem216} sometimes contains an expression of the form $\Gamma(-n),\, n \in \N$ which in principle leads to undefined behaviour (cf. \Cref{def:gamma-function} together with the property that $k \Gamma(k) = \Gamma(k+1)$).
However, one can consider the limit as $x \in \R^+$ approaches an integer $n$ to say that
$$\lim_{x \goesto n} \Gamma(-x) = (-1)^{n-1} \infty\,, \quad\text{or equivalently}\quad \lim_{x \goesto n} \frac{1}{\Gamma(-x)} = 0\,,$$
in which case we are lucky because $\Gamma(\beta/2-n+1)$ appears in the denominator of the prefactor and without any singularities in the numerator we can safely evaluate the entire expression to $0$.

Because $I_{m,n}^{\alpha,\beta}(\vec{x})$ only depends on the squared radius $r^2 = \norm{\vec{x}}^2$, let it also be denoted by $I_{m,n}^{\alpha,\beta}(r^2) = I_{m,n}^{\alpha,\beta}(\vec{x})$.
Further, let
\begin{equation}
  \bar{I}_{m,n}^{\alpha,\beta} := \frac{\left\langle x \mapsto I\left(\frac{x+1}{2}\right), P_0^{(a, b)} \right\rangle}{\left\langle P_0^{(a, b)}, P_0^{(a, b)} \right\rangle} = \frac{1}{h_0^{(a, b)}} \int_{-1}^{1} I_{m,n}^{\alpha,\beta}\left(\frac{x+1}{2}\right) w(x) \,\ddx\,,
  \label{eq:0th-coeff-of-I}
\end{equation}
denote the 0th coefficient in a Jacobi expansion of $I_{m,n}^{\alpha,\beta}(\vec{x})$ (recall that $r^2 = \frac{x+1}{2}$ and $P_0^{(a, b)}(x) = 1$) with $h_k := \left\langle P_k^{(a, b)}, P_k^{(a, b)} \right\rangle$ the normalisation coefficients of the basis.

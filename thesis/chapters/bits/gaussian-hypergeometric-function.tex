\begin{lemma}{Gaussian Hypergeometric Function}{gaussian-hypergeometric-function}
  The $p=2$, $q=1$ special case of the generalised hypergeometric series can also be evaluated by
  $${}_2F_1\left(\begin{matrix}a_{1}, -n \\b_{1}\end{matrix}; z\right) = \sum_{k=0}^n (-1)^k \binom{n}{k} \frac{(a_1)_k}{(b_1)_k}z^k\,,$$
  when the second argument $a_2 = -n$ is a non-positive integer, so $n \in \N_0$.
\end{lemma}
\begin{proof}
  Starting from the definition of the generalised hypergeometric series ${}_pF_q$ with $p=2$ and $q=1$ (\Cref{def:generalised-hypergeometric-series}),
  $${}_2F_1\left(\begin{matrix}a_{1}, -n \\b_{1}\end{matrix}; z\right) = \sum_{k=0}^{\infty} \frac{(a_1)_k (-n)_k}{(b_1)_k} \frac{z^k}{k!} = \sum_{k=0}^{n} \frac{(a_1)_k (-n)_k}{(b_1)_k} \frac{z^k}{k!}\,,$$
  which can be terminated at $k=n$ due to \Cref{remark:non-positive-pochhammer}, we can express
  $$\frac{(-n)_k}{k!} = \binom{-n+k-1}{k} = (-1)^k \binom{1+n-k+k-1}{k} = (-1)^k \binom{n}{k}$$
  using a well-known relation between the Pochhammer symbol and the binomial coefficient \parencite{2001-pochhammer-binomial-relation,2018-nist} which immediately leads us to
  $${}_2F_1\left(\begin{matrix}a_{1}, -n \\b_{1}\end{matrix}; z\right) = \sum_{k=0}^n \binom{n}{k} \frac{(a_1)_k}{(b_1)_k}(-z)^k\,,$$
  concluding the proof.
\end{proof}

Note that these functions are generally tricky to evaluate efficiently, only recent advancements have enabled their usage in a broader range of applications \parencite{2008-hypergeometric-functions-jl-1,2017-hypergeometric-functions-jl-2,2023-hypergeometric-functions-jl-3}.
Implementations are available in the \href{https://github.com/JuliaMath/HypergeometricFunctions.jl}{HypergeometricFunctions.jl} package in Julia.

More details on the Gaussian hypergeometric series, sometimes simply referred to as the hypergeometric function, its defining differential equation origin, modular interpretations and symmetries can be found in the 1997 book \citetitle{1997-hypergeometric-functions-my-love} \parencite{1997-hypergeometric-functions-my-love}.

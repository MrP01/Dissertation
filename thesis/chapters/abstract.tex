\chapter*{Abstract}
\label{chap:abstract}
What density distribution does a system of particles, a flock of birds or a school of fish approach in equilibrium?
In this dissertation, we explore the construction of a spectral method for the solution of these particle density distributions $\hat{\rho}$, also referred to as \textit{equilibrium measures}, on a $d$-dimensional ball of radius $R$.
By modelling their interaction through a pairwise interaction potential $K(r)$ only dependent on the distance $r$ between two entities in the system, and a self-propulsion term in the case of animals (so-called \textit{active matter}), our description generalises from animals to physical particles (atoms and molecules).
The spectral method is constructed using radial Jacobi polynomials $P_k^{(a,b)}$, which are orthogonal with respect to a certain weight function.
They provide the basis for an $N$-term expansion of the solution and its coefficients are obtained through the solution of a linear system.
By construction, the operators that appear are approximately banded and sparse, leading to a highly effective direct method for the solution of equilibrium measures to arbitrary precision.

\paragraph{Statement of Originality:}
The extension of the attractive-repulsive kernel spectral method into a general kernel spectral method along with an implementation of it is original.
All code contributions, starting from the particle simulation software to the implementation of the spectral methods, are entirely original.
We present a lemma for exact calculation of the support radius $R$ when using an $N = 1$ order approximation of the solution and show that it is a unique minimiser.
For higher orders, this result provides a solid initial guess for a subsequent optimisation routine.

\paragraph{Keywords:}
\thesiskeywords

\paragraph{Implemented using:}
C++, Julia, Python

\chapter{Introduction}
\label{chap:introduction}

Let $\N$ denote the natural numbers (positive integers) without $0$ and let $\N_0 := \N \cup \{0\}$.
In the following, we will use \textbf{bold} notation for vectors, matrices will generally be denoted by a capital letter and scalars by a lowercase letter.
We will frequently make use of the (Euclidean) 2-norm of a vector, as denoted by $\norm{\cdot}_2$ or simply $\norm{\cdot}$.
So for a $d$-dimensional vector $\vec{x} \in \R^d$ we have $\norm{\vec{x}} := \sqrt{\sum_{k=1}^d x_k^2}$.

One should also clarify the nature of a few of the integrals appearing in this thesis which are often performed over the closed unit ball $B_1(\vec{x}) := \{\vec{y} \in \R^d \;|\, \norm{\vec{x} - \vec{y}} \le 1\}$ centered at the origin $\vec{x} = \vec{0}$.
These volume integrals (often ended by $\dd^d y$ or $\dd V$) over the $d$-dimensional unit ball shall be written as
$$\int_{B_1(\vec{0})} \dd\vec{y}\,,$$
where $\vec{y} \in \R^d$ is the integration variable.
Note that some definitions of $B_1(\vec{x})$ are open sets, leaving out the shell $\{\vec{y} \in \R^d \;|\, \norm{\vec{x} - \vec{y}} = 1\}$.
The choice of definition does not matter for our purposes as the shell, a hyperplane of Lebesgue measure $0$, does not contribute to the integral.

\begin{figure}[H]
  \centering
  \begin{subfigure}[t]{0.5\textwidth}
    \centering
    \inputtikz{problem-setting}
    \caption{$N = 8$ particles interacting with one another through the potential $K(r)$.}
    \label{fig:problem-setting}
  \end{subfigure}
  \hfill
  \begin{subfigure}[t]{0.49\textwidth}
    \centering
    \scalebox{0.68}{\inputtikz{potential-function}}
    \caption{Plot of attractive-repulsive potential functions $K(r) = \frac{r^\alpha}{\alpha} - \frac{r^\beta}{\beta}$ for different $\alpha, \beta$.}
    \label{fig:potential-function}
  \end{subfigure}
\end{figure}

Cf. \Cref{fig:problem-setting} and \Cref{fig:potential-function}.

All plots and figures in this thesis were generated using the Makie visualisation tool \parencite{2021-makie}, an open-source package available for the Julia computing language \parencite{2017-julia}.

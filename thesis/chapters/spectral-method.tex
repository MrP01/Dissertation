\chapter{Spectral Method}
\label{chap:spectral-method}

% \section{Content}
% \input{chapters/out/Spectral Method.md.tex}

\section{Definitions}
\begin{definition}{Rising Factorial}{rising-factorial}
  \begin{center}\rule{0.5\linewidth}{0.5pt}\end{center}

  \hypertarget{alias-pochhammer-symbol}{%
    \subsection{alias: Pochhammer Symbol}\label{alias-pochhammer-symbol}}

  Given by \[(x)_n = \prod_{k=0}^{n-1} (x+k)\,.\]
\end{definition}

\begin{definition}{Orthogonal Polynomials}{orthogonal-polynomials}
  Are univariate polynomials
  \[p: \mathbb{R} \mapsto \mathbb{R}, \; p(x) = \sum_{k=1}^N c_k x^k\,.\]
  that form an orthogonal basis under some inner product.
\end{definition}

\begin{theorem}{Three-Term Recurrence Relationship}{three-term-recurrence-relationship}
  All orthogonal polynomials $\{p_0, p_1, p_2, ...\}$ (cf. \Cref{def:orthogonal-polynomials}) have (at least) a three-term recurrence relationship of the form
  $$A_n p_{n+1}(x) = (B_n - x) p_n(x) + C_n p_{n-1}(x)\,.$$
\end{theorem}
\begin{proof}
  Consider ``$x p_n$''$:= x \mapsto x p_n(x)$, a polynomial with $\deg(x p_n) \le n+1$.
  By the linear independence of all orthogonal polynomials $p_n$ with respect to the inner product $\langle \cdot, \cdot \rangle_w$, it must be possible to write
  $$x p_n(x) = \sum_{k=0}^{n+1} \hat{a}_k p_k(x)\,,\quad \text{for some}~\hat{a}_k \in \R, k=0, ..., n+1\,.$$
  Now, for all $n \ge 0$ and $m \le n+1$ we have
  $$\langle xp_n, p_m \rangle_w = \sum_{k=0}^{n+1} \hat{a}_k \langle p_k, p_m \rangle_w = \sum_{k=0}^{n+1} \hat{a}_k \delta_{i,k} = \hat{a}_m \langle p_m, p_m \rangle_w\,,$$
  due to the orthogonality relationship (\Cref{thm:jacobi-orthogonality-condition}).
  Therefore,
  \begin{equation}
    \hat{a}_m = \frac{\langle xp_n, p_m \rangle_w}{\langle p_m, p_m \rangle_w} \quad\text{for all}~m \le n+1\,.
    \label{eq:three-term-step}
  \end{equation}

  But when \underline{$m < n-1$}, we have $\deg(xp_m) < n$ so $x p_m(x) = \sum_{k=0}^{n-1} \hat{b}_k p_k(x)$ for some (potentially 0) $\hat{b}_k \in \R$, and therefore $\langle p_n, xp_m \rangle_w = \sum_{k=0}^{n-1} \hat{b}_k \langle p_n, p_k \rangle_w = 0$,
  which, by the symmetry of the inner product (\Cref{remark:symmetry-of-inner-product}), also implies $\langle x p_n, p_m \rangle_w = 0$ which, by \Cref{eq:three-term-step}, allows us to conclude that the earlier coefficients $\hat{a}_m = 0$.

  % And obviously, even if we assumed higher-order coefficients in the expansion of $xp_n$, when $m > n+1 \Leftrightarrow n < m-1$, $\deg(xp_n) < m$ and so all later coefficients $\hat{a}_m = 0$.

  We recall that $xp_n(x) = \sum_{k=0}^{n+1} \hat{a}_k p_k(x)$, which in combination with our insights on the $\hat{a}_m$ above means that
  $$xp_n(x) = \hat{a}_{n-1} p_{n-1}(x) + \hat{a}_{n} p_n(x) + \hat{a}_{n+1} p_{n+1}(x)\,,$$
  concluding the proof.
\end{proof}

For example, for the Chebyshev polynomials $T_k: [-1, 1] \mapsto \R$ we have
$$T_{k+1}(x) = 2x T_k(x) - T_{k-1}(x) \,.$$

Note that the converse of \Cref{thm:three-term-recurrence-relationship} is also true, a set of polynomials of increasing degree $k$ that has a three-term recurrence relationship is a set of orthogonal polynomials (cf. \Cref{def:orthogonal-polynomials}).
While the original theorem was discovered before \cite{1935-favard}, to this day we still refer to it as \textit{Favard's theorem}.

% From HeatFun:
% \begin{theorem}{Chebyshev recursion formula}{chebrecursion}
%   The \chebyshev polyomials satisfy the three-term recurrence relation $$T_{k+1}(x) = 2x T_k(x) - T_{k-1}(x) \,.$$
% \end{theorem}
% \begin{proof}{\parencite{atap}.}
%   For $k > 1$, we have
%   \begin{align*}
%     2x T_k(x) - T_{k-1}(x) & = 2x \cdot \frac{1}{2} (z^k + z^{-k}) - \frac{1}{2} (z^{k-1} + z^{-(k-1)})                     \\
%                            & = 2 \frac{1}{2}(z + z^{-1}) \cdot \frac{1}{2}(z^k + z^{-k}) - \frac{1}{2} (z^{k-1} + z^{-k+1}) \\
%                            & = \frac{1}{2} (z^{k+1} + z^{k-1} + z^{-k+1} + z^{-k-1}) - \frac{1}{2} (z^{k-1} + z^{-k+1})     \\
%                            & = \frac{1}{2} (z^{k+1} + z^{-(k+1)}) = T_{k+1}(x)
%   \end{align*}
% \end{proof}

\begin{definition}{Generalised Hypergeometric Series}{generalised-hypergeometric-series}
  Is given by \[_pF_q\] Special Case: {[}{[}Gaussian Hypergeometric
  Function{]}{]}. The definition involves the Rising Factorial (cf. \Cref{def:rising-factorial})
  (Pochhammer Symbol).
  $${ \,{}_{p}F_{q}(a_{1},\ldots ,a_{p};b_{1},\ldots ,b_{q};z) := \sum _{n=0}^{\infty }{\frac {(a_{1})_{n}\cdots (a_{p})_{n}}{(b_{1})_{n}\cdots (b_{q})_{n}}}\,{\frac {z^{n}}{n!}}.}\,.$$
\end{definition}

\begin{definition}{Gaussian Hypergeometric Function}{gaussian-hypergeometric-function}
  Written as \[_2F_1(a, b; c; z)\]
\end{definition}

\begin{definition}{Jacobi Polynomials}{jacobi-polynomials}
  Are given by \[J^{(a,b)}_n(x) = \mathrm{prefactor} \cdot {}_2F_1(...)\]
  So are defined using the {[}{[}Gaussian Hypergeometric Function{]}{]}.

  \hypertarget{nice-spectral-properties}{%
    \subsection{Nice Spectral Properties}\label{nice-spectral-properties}}

  \begin{itemize}
    \item
          Differentiation
    \item
          Three-Term Recurrence
    \item
          {[} {]} why are they better than just Chebyshev?
  \end{itemize}

  {[}{[}Gegenbauer Polynomials{]}{]} are a special case. And
  {[}{[}Chebyshev Polynomials{]}{]} are a special case of them.
\end{definition}

\begin{definition}{Gegenbauer Polynomials}{gegenbauer-polynomials}
  \hypertarget{alias-ultraspherical-polynomials}
  {alias: Ultraspherical Polynomials}

  Are a special case of the Jacobi Polynomials (cf. \Cref{def:jacobi-polynomials}) and form an
  Orthonormal Basis (cf. \Cref{def:orthonormal-basis}) under the weight given by
  \[w(x)=(1+x)^\alpha\]
\end{definition}

\begin{definition}{Chebyshev Polynomials}{chebyshev-polynomials}
  Of the first kind: \[T_k(x)\] Of the second kind: \[U_k(x)\] Also have a
  {[}{[}Three-Term Recurrence Relationship{]}{]}.
\end{definition}

\begin{definition}{Jacobi Matrix}{jacobi-matrix}
  aliases: Jacobi Operator

  The \href{https://en.wikipedia.org/wiki/Jacobi_operator}{Jacobi
    operator} is the matrix \(X \in \R^{N \times N}\) satisfying
  $$x \cdot P(x) = P(x) \cdot X^T$$
\end{definition}

\begin{definition}{Ansatz}{ansatz}
  \[\rho(x) = (1-||{y}||^{2})^{m - \frac{\alpha + d}{2}} \sum_{k=1}^{N} P_{k}^{(a,b)}(2 ||y||^{2}-1)\]

  Todo: - {[} {]} is it alpha or beta in the exponent of (1-y\^{}2)?
\end{definition}

Similar to \Cref{def:power-law-potential}, we can define the (single) power law operator $\mathcal{Q}^\beta$:

% TODO: explain why we are not solving \int\int but only the following
\begin{definition}{Power Law Operator $\mathcal{Q}^\beta$}{power-law-operator}
  The power law operator $\mathcal{Q}^\beta: \functionspace \mapsto \functionspace$ is given by
  $$\mathcal{Q}^\beta[\rho](\vec{x}) := \int \norm{\vec{x}-\vec{y}}^\beta \,\dd\rho(\vec{y}) = \int_{\supp(\rho)} \norm{\vec{x}-\vec{y}}^\beta \rho(\vec{y}) \,\dd\vec{y}\,,$$
\end{definition}
% Either the attractive or the repulsive operator can be sparse.
% Obtained using \Cref{thm:theorem216}.

The operator $\mathcal{Q}^\beta$ acting on an equilibrium measure $\rho(\vec{x})$ returns the energy $\tilde{E}(\vec{x}) = \mathcal{Q}^\beta[\rho](\vec{x})$ at a point $\vec{x} \in B_1(\vec{0})$ in our normalised domain.

\begin{definition}{Spectral Convergence}{spectral-convergence}
  An \(N\)-point approximation \(\varphi_N\) of a function f converges to \(f\) at spectral speed if \(|\varphi_N -f|\) decays pointwise in \([-1, 1]\) faster than \(\mathcal{O}(N^{-p})\) for any \(p = 1, 2, . . .\) so \(p \in \mathbb{N}\).
\end{definition}


\begin{theorem}{Integration Theorem that needs a name}{theorem216}
  On the $d$-dimensional unit ball $B_1$ the power law potential, with power $\alpha \in(-d,2+2m-d)$, $m\in\mathbb{N}_0$ and $\beta>-d$, of the $n$-th weighted radial Jacobi polynomial $$(1-|y|^2)^{m-\frac{\alpha+d}{2}}P_n^{(m-\frac{\alpha+d}{2},\frac{d-2}{2})}(2|y|^2-1)$$ reduces to a Gaussian hypergeometric function as follows:
  \begin{align*}
    \int_{B_1} & |x-y|^\beta (1-|y|^2)^{m-\frac{\alpha+d}{2}} P_n^{(m-\frac{\alpha+d}{2},\frac{d-2}{2})}(2|y|^2-1) \mathrm{d}y                                                                                                                                                                                                                                                                                               \\
               & = \tfrac{\pi ^{d/2} \Gamma \left(1+\frac{\beta}{2}\right) \Gamma \left(\frac{\beta+d}{2}\right) \Gamma \left(m+n-\frac{\alpha+d}{2}+1\right)}{\Gamma \left(\frac{d}{2}\right) \Gamma (n+1) \Gamma \left(\frac{\beta}{2}-n+1\right) \Gamma \left(\frac{\beta-\alpha}{2}+m+n+1\right)}{}_2F_1\left(\begin{matrix}n-\frac{\beta}{2}, \quad -m-n+\frac{\alpha-\beta}{2} \\\frac{d}{2}\end{matrix};|x|^2\right).
  \end{align*}
\end{theorem}

\Cref{thm:theorem216} gives an explicit expression for the main integral
\(Q^{\beta}: L \mapsto L\), an operator from the Function Space \(L\) to the function space \(L\), we are interested in:
\[\hat{Q}^{\beta}[\rho](x) = \int_{B_1} |x-y|^\beta (1-|y|^2)^{m-\frac{\alpha+d}{2}} P_n^{(m-
  \frac{\alpha+d}{2},\frac{d-2}{2})}(2|y|^2-1) \mathrm{d}y\] which is used
to construct the Spectral Method Operator \(Q^\beta\) (cf. \Cref{def:operator}), acting on the coefficients \(\vec{\rho}\).

\section{Derivation of Operator}
We start by only considering a single power law operator (out of two in the case of an attractive-repulsive interaction potential $K_{\alpha, \beta}$).
Substituting our ansatz given in \Cref{eq:ansatz} into $\mathcal{Q}^\beta[\rho]$, we obtain
\begin{equation}
  \mathcal{Q}^{\beta}[\rho](x) = \sum_{k=0}^{N-1} \rho_{k} \int_{B_1(\vec{0})} \norm{\vec{x}-\vec{y}}^{\beta} \weight{y}\jacobi{y} \,\dd\vec{y}\,,
  \label{eq:theorem216-in-derivation}
\end{equation}
luckily containing the integral evaluated in \Cref{thm:theorem216}.

We are now interested in a numerical representation of the operator $\mathcal{Q}^\beta$ acting on the function $\rho \in \functionspace$, so an equivalent (linear) operator $Q^\beta: \R^N \mapsto \R^N$ acting on the coefficients $\rho_k \in \R,\, k = 0, ..., N-1$.
As every finite-dimensional linear operator must have a matrix representation, we look for a $Q^\beta \in \R^{N \times N}$ such that
$$\mathcal{Q}^\beta[\rho](\vec{x}) = \jacobivec{x} \cdot Q^\beta \vec{\rho}\,,$$
where $\jacobivec{x} \in \R^N$ is the vector of radial Jacobi polynomials $P^{(a, b)}_0(x)$, $P^{(a, b)}_1(x)$, ..., $P^{(a, b)}_{N-1}(x)$ evaluated at $2\norm{\vec{x}}^2 - 1$ as introduced in and after \Cref{def:jacobi-polynomials}.
Note that in the context of linear combinations of Jacobi polynomials, we will use zero-based indexing for vectors and matrices due to the convention that the first orthogonal polynomial is usually denoted by $p_0(x) = 1$, in line with $\deg(p_k) = k$.

Based on the three-term recurrence relationship (cf. \Cref{thm:three-term-recurrence-relationship}), one can even determine an explicit relationship between the coefficients in the Jacobi expansion by considering the Jacobi matrix (cf. \Cref{remark:jacobi-matrix}).
We use this recurrence relationship in our implementation to significantly speed up the construction of the operator (cf. \Cref{sec:runtime-analysis}).
The recurrence coefficients used are due to \cite{2021-arbitrary-dimensions} and \cite{2023-olver-equilibrium-measures-jl}.

% TODO: this section has weird indexing (start at 0 or 1?) and needs some work.
Therefore, starting from \Cref{eq:theorem216-in-derivation}, we obtain
\begin{align*}
  \mathcal{Q}^\beta[\rho](\vec{x}) & = \sum_{k=0}^{N-1} \rho_k \mathcal{Q}^\beta[wP_k](\vec{x}) = \sum_{k=0}^{N-1} \rho_k \sum_{j=0}^{N-1} q^{\beta}_{kj} \jacobi{x} \\
                                   & = \sum_{j=0}^{N-1} \sum_{k=0}^{N-1} \rho_k q^{\beta}_{kj} \jacobi{x}\,,
\end{align*}
which we will rewrite in matrix-form,
\begin{align*}
  \mathcal{Q}^\beta[\rho](\vec{x}) & = \vec{P}(\vec{x}) \cdot
  \begin{pmatrix}
    \sum_{k=0}^{N-1} \rho_k q^{\beta}_{k,1} \\
    \vdots                                  \\
    \sum_{k=0}^{N-1} \rho_k q^{\beta}_{k,N}
  \end{pmatrix} = \vec{P}(\vec{x}) \cdot
  \underbrace{
    \begin{pmatrix}
      q^{\beta}_{00}    & \dots  & q^{\beta}_{0,N-1}   \\
      \vdots            & \ddots & \vdots              \\
      q^{\beta}_{N-1,0} & \dots  & q^{\beta}_{N-1,N-1} \\
    \end{pmatrix}
  }_{=: Q^\beta}
  \begin{pmatrix}
    \rho_0 \\
    \vdots \\
    \rho_{N-1}
  \end{pmatrix}                                                              \\
                                   & = \jacobivec{x} \cdot Q^\beta \vec{\rho}
\end{align*}
where we use $\vec{P}(\vec{x}) = \jacobivec{x}$ as a shorthand giving us the form of the operator matrix.
Its first row, the set of coefficients for the constant Jacobi polynomial $P_0^{(a,b)}$ of each expansion of $Q^\beta$, weighted by the solution coefficients, adds up to the total energy $E$ (coefficient of the constant polynomial).
All remaining rows of $Q^\beta$ contain coefficients for polynomials of degree at least 1, so their weighted contributions must add up to 0 in order for the total energy $\tilde{E}(\vec{x})$ to remain constant.

For the attractive-repulsive interaction potential $K_{\alpha,\beta}$, because each column consists of an expansion of the Jacobi polynomials, we have $q^{\beta}_{0k} = I_{m,k}^{\alpha,\beta}$.

\begin{figure}[H]
  \centering
  \includegraphics[width=\linewidth]{results/attrep/attractive-repulsive-operators.pdf}
  \caption[Attractive and repulsive operators.]{The attractive and repulsive operators (matrices) as given in \Cref{def:power-law-operator}, the matrix values are shown in a $\log_{10}$ colour scale. Due to the choice of basis, the attractive operator is exactly banded. The repulsive parameter is only approximately banded, which the spy plots effectively demonstrate.}
  \label{fig:attractive-repulsive}
\end{figure}

The bandedness of the attractive operator in \Cref{fig:attractive-repulsive} is due to the three-term recurrence relationship of the Jacobi polynomial basis (cf. \Cref{thm:three-term-recurrence-relationship} and \Cref{def:jacobi-polynomials}).
% TODO.. explain


\begin{figure}[H]
  \centering
  \label{fig:attractive-repulsive}
  \includegraphics[width=\linewidth]{results/attractive-repulsive-operators.pdf}
  \caption[Attractive and repulsive operators.]{The attractive and repulsive operators (matrices) as given in \Cref{def:operator}, the matrix values are shown in a $\log_{10}$ color scale. Due to the choice of basis, the attractive operator is exactly banded. The repulsive parameter is only approximately banded, which the spy plots effectively demonstrate.}
\end{figure}

The bandedness of the attractive operator is due to the three-term recurrence relationship of the Jacobi polynomial basis.
% TODO.. explain

For the attractive-repulsive interaction potential, the full operator is given by
\begin{equation}
  Q_{\alpha, \beta} := \frac{R^\alpha}{\alpha} Q^\alpha - \frac{R^\beta}{\beta} Q^\beta
  \label{eq:full-attrep-operator}
\end{equation}
for some interval radius $R \in \R^+$, usually chosen as the smallest possible $R$ such that $\supp(\rho) \subseteq [-R, R]$.

\begin{figure}[H]
  \centering
  \label{fig:attrep-operator}
  \includegraphics[width=0.5\linewidth]{results/attrep/full-operator.pdf}
  \caption[Combination of the attractive-repulsive operators]{Spy plot of $Q_{\alpha, \beta}$, the combination of the attractive-repulsive operators. Inverting this operator and applying it to $(1, 0, ..., 0)^T \in \R^N$ will yield the unnormalised coefficients $\rho_k$ of the solution expansion given in \Cref{def:ansatz}.}
\end{figure}

\section{Results}
\begin{figure}[H]
  \centering
  \label{fig:solution-increasing-order}
  \includegraphics[width=0.8\linewidth]{results/attrep/solution-increasing-order.pdf}
  \caption[Solutions of increasing orders]{Particle density distribution function solutions $\rho$ of increasing order $N$ to the attractive-repulsive problem with interaction potential $K_{alpha, \beta}(r)$, $\alpha = 2.5$ and $\beta = 1.2$. Reflected along the y-axis for better visibility of the domain.}
\end{figure}

\section{Outer Optimisation Routine}
\begin{figure}[H]
  \centering
  \label{fig:outer-optimisation}
  \includegraphics[width=0.8\linewidth]{results/known-analytic/outer-optimisation.pdf}
  \caption[Outer Optimisation Routine]{The total potential $U$ as a function of the support radius $R$. This is the goal function minimised by the outer optimisation routine.}
\end{figure}

Note that using this setup, the operators themselves do not need to be recomputed (cf. \Cref{eq:full-attrep-operator}).
The provided implementation uses \gls{lru} caching to automatically store operators for a given parameter set and order $N$.

\section{Comparison with Analytic Solutions}
As introduced in \Cref{sec:analytical-solutions}, there are some analytical solutions available which allow us to perform some further analysis of the numerical method in these special cases.

\begin{figure}[H]
  \centering
  \label{fig:analytic-solution}
  \includegraphics[width=\linewidth]{results/analytic-solution.pdf}
  \caption[Comparison with analytical solutions and error]{
    The analytic solution $\rho(x)$ given in \Cref{eq:analytical-solution-alpha-equal-2} compared to the (spectral method) solutions of different order $N$.
    The ``arches'' occur as a result of the roots of $\rho(x) - \rho_N(x)$, their number equals the order $N$ (a polynomial of order $N$ has $N$ roots).
  }
\end{figure}

\begin{figure}[H]
  \centering
  \label{fig:convergence-to-analytic}
  \includegraphics[width=0.6\linewidth]{results/convergence-to-analytic.pdf}
  \caption[Convergence to analytic solution]{Convergence of the numerical solution to the known analytic solution (cf. \Cref{eq:analytical-solution-alpha-equal-2}) in a special case where it is known, squared error plotted as a function of the highest order in the expansion $N$.}
\end{figure}

\section{Discussion}
\begin{figure}[H]
  \centering
  \label{fig:convergence}
  \includegraphics[width=0.7\linewidth]{results/convergence.pdf}
  \caption[Step-by-step convergence of solutions compared to order 24]{Step-by-step convergence of numerical solutions $\rho_N(x)$ as compared to $\rho_{24}(x)$, visualised using the squared error of the pointwise evaluation of both functions in $200$ points.}
\end{figure}

\begin{figure}[H]
  \centering
  \label{fig:spatial-energy-dependence}
  \includegraphics[width=0.6\linewidth]{results/energy-dependence-on-r.pdf}
  \caption[Spatial energy dependence on $r$]{Plot of the spatial energy dependence on $r$, for different values of the domain support radius $R$. As one can see, they are constant and this figure is only present as visual proof to increase our confidence in the construction of the spectral method.}
\end{figure}

\begin{figure}[H]
  \centering
  \label{fig:varying-parameters}
  \includegraphics[width=0.9\linewidth]{results/varying-parameters.pdf}
  \caption[Varying parameters in the solver]{
    Varying different parameters in the solver to demonstrate their effect.
    See also, \Cref{fig:varying-R-solutions}.
  }
\end{figure}

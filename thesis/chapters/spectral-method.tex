\chapter{Spectral Method}
\label{chap:spectral-method}
In this chapter we will construct a spectral method in the basis of Jacobi polynomials to explore the solution of equilibrium distributions $\hat{\rho}(\hatvec{x})$.
Starting from a many-body system and considering the continuous limit as $N_p \goesto \infty$, in \Cref{chap:particle-interaction-theory} we have already established the governing equation of the particle density distribution $\hat{\rho}(\hatvec{x})$ in such a system.

As mentioned in \Cref{chap:particle-interaction-theory}, the numerical approaches will be carried out on the normalised domain $B_1(\vec{0})$ and we will be looking for $\rho \in \functionspace$, whilst in general we are interested in the $\hat{\rho} \in \functionspacehat$ solving \href{def:the-problem}{our problem}.
Both versions are related by $\hat{\rho}(\hatvec{x}) = \rho(\vec{x})$ and $\hatvec{x} := R \vec{x}$.

Can we put together a numerical method to solve for the equilibrium distribution (cf. \Cref{def:equilibrium-measure})?
Let us consider the problem from the bottom up and start from the solution:
The basic idea behind spectral methods is to assume a solution $\rho(\vec{x})$ of the form
$$\rho(\vec{x}) = \sum_{k=0}^{N-1} \rho_k \varphi_k(\vec{x})\,,\quad \rho_k \in \R, \varphi_k: \R^d \mapsto \R\,,\quad k = 0, ..., N-1\,,$$
with $N$ coefficients $\vec{\rho} := \left(\rho_0, ..., \rho_{N-1}\right)^T$ multiplying $N$ basis functions $\varphi_k$.

\pagebreak
\section{Special Functions}
The following section will introduce a few necessary objects and tools to understand the basis of functions we are working with to construct the spectral method, the basis of Jacobi polynomials.

We start with the Pochhammer symbol, another name for the \textit{rising factorial}, an unusual notation for a function but standard in the context of the special functions that will be introduced on top of it.
\begin{definition}{Rising Factorial}{rising-factorial}
  \begin{center}\rule{0.5\linewidth}{0.5pt}\end{center}

  \hypertarget{alias-pochhammer-symbol}{%
    \subsection{alias: Pochhammer Symbol}\label{alias-pochhammer-symbol}}

  Given by \[(x)_n = \prod_{k=0}^{n-1} (x+k)\,.\]
\end{definition}


As a second prerequisite, we introduce the closely intertwined beta- and gamma-functions (\Cref{def:beta-function}, \Cref{def:gamma-function}).
\begin{definition}{Gamma Function}{gamma-function}
  Aligning with the factorial for integer arguments, $\Gamma: \R^+ \mapsto \R$ is given by
  $$\Gamma(x) := \int_0^\infty t^{x-1}\e^{-t} \ddt\,.$$
\end{definition}

\begin{definition}{Beta Function}{beta-function}
  $B: \R^+\times\R^+ \mapsto \R$ is given by
  $$B(x_1, x_2) := \int_0^1 t^{x_1-1}(1-t)^{x_2-1} \ddt\,.$$
\end{definition}

Note that following from this definition, there is a relationship with the gamma-function
$$B(x_1, x_2) = \frac{\Gamma(x_1) \Gamma(x_2)}{\Gamma(x_1 + x_2)}\,.$$


Using the Pochhammer symbol introduced in \Cref{def:rising-factorial}, we can now define the generalised hypergeometric series ${}_pF_q$ (cf. \Cref{def:generalised-hypergeometric-series}) and a special case of it, the Gaussian hypergeometric function (cf. \Cref{lemma:gaussian-hypergeometric-function}).
\begin{definition}{Generalised Hypergeometric Series}{generalised-hypergeometric-series}
  Is given by \[_pF_q\] Special Case: {[}{[}Gaussian Hypergeometric
  Function{]}{]}. The definition involves the Rising Factorial (cf. \Cref{def:rising-factorial})
  (Pochhammer Symbol).
  $${ \,{}_{p}F_{q}(a_{1},\ldots ,a_{p};b_{1},\ldots ,b_{q};z) := \sum _{n=0}^{\infty }{\frac {(a_{1})_{n}\cdots (a_{p})_{n}}{(b_{1})_{n}\cdots (b_{q})_{n}}}\,{\frac {z^{n}}{n!}}.}\,.$$
\end{definition}

\begin{definition}{Gaussian Hypergeometric Function}{gaussian-hypergeometric-function}
  Written as \[_2F_1(a, b; c; z)\]
\end{definition}


\pagebreak
\section{Orthogonal Polynomials Forming a Basis}
In order to efficiently construct a spectral method, we need an orthogonal basis.
\begin{definition}{Orthogonal Polynomials}{orthogonal-polynomials}
  Are univariate polynomials
  \[p: \mathbb{R} \mapsto \mathbb{R}, \; p(x) = \sum_{k=1}^N c_k x^k\,.\]
  that form an orthogonal basis under some inner product.
\end{definition}

\begin{theorem}{Three-Term Recurrence Relationship}{three-term-recurrence-relationship}
  All orthogonal polynomials $\{p_0, p_1, p_2, ...\}$ (cf. \Cref{def:orthogonal-polynomials}) have (at least) a three-term recurrence relationship of the form
  $$A_n p_{n+1}(x) = (B_n - x) p_n(x) + C_n p_{n-1}(x)\,.$$
\end{theorem}
\begin{proof}
  Consider ``$x p_n$''$:= x \mapsto x p_n(x)$, a polynomial with $\deg(x p_n) \le n+1$.
  By the linear independence of all orthogonal polynomials $p_n$ with respect to the inner product $\langle \cdot, \cdot \rangle_w$, it must be possible to write
  $$x p_n(x) = \sum_{k=0}^{n+1} \hat{a}_k p_k(x)\,,\quad \text{for some}~\hat{a}_k \in \R, k=0, ..., n+1\,.$$
  Now, for all $n \ge 0$ and $m \le n+1$ we have
  $$\langle xp_n, p_m \rangle_w = \sum_{k=0}^{n+1} \hat{a}_k \langle p_k, p_m \rangle_w = \sum_{k=0}^{n+1} \hat{a}_k \delta_{i,k} = \hat{a}_m \langle p_m, p_m \rangle_w\,,$$
  due to the orthogonality relationship (\Cref{thm:jacobi-orthogonality-condition}).
  Therefore,
  \begin{equation}
    \hat{a}_m = \frac{\langle xp_n, p_m \rangle_w}{\langle p_m, p_m \rangle_w} \quad\text{for all}~m \le n+1\,.
    \label{eq:three-term-step}
  \end{equation}

  But when \underline{$m < n-1$}, we have $\deg(xp_m) < n$ so $x p_m(x) = \sum_{k=0}^{n-1} \hat{b}_k p_k(x)$ for some (potentially 0) $\hat{b}_k \in \R$, and therefore $\langle p_n, xp_m \rangle_w = \sum_{k=0}^{n-1} \hat{b}_k \langle p_n, p_k \rangle_w = 0$,
  which, by the symmetry of the inner product (\Cref{remark:symmetry-of-inner-product}), also implies $\langle x p_n, p_m \rangle_w = 0$ which, by \Cref{eq:three-term-step}, allows us to conclude that the earlier coefficients $\hat{a}_m = 0$.

  % And obviously, even if we assumed higher-order coefficients in the expansion of $xp_n$, when $m > n+1 \Leftrightarrow n < m-1$, $\deg(xp_n) < m$ and so all later coefficients $\hat{a}_m = 0$.

  We recall that $xp_n(x) = \sum_{k=0}^{n+1} \hat{a}_k p_k(x)$, which in combination with our insights on the $\hat{a}_m$ above means that
  $$xp_n(x) = \hat{a}_{n-1} p_{n-1}(x) + \hat{a}_{n} p_n(x) + \hat{a}_{n+1} p_{n+1}(x)\,,$$
  concluding the proof.
\end{proof}

For example, for the Chebyshev polynomials $T_k: [-1, 1] \mapsto \R$ we have
$$T_{k+1}(x) = 2x T_k(x) - T_{k-1}(x) \,.$$

Note that the converse of \Cref{thm:three-term-recurrence-relationship} is also true, a set of polynomials of increasing degree $k$ that has a three-term recurrence relationship is a set of orthogonal polynomials (cf. \Cref{def:orthogonal-polynomials}).
While the original theorem was discovered before \cite{1935-favard}, to this day we still refer to it as \textit{Favard's theorem}.

% From HeatFun:
% \begin{theorem}{Chebyshev recursion formula}{chebrecursion}
%   The \chebyshev polyomials satisfy the three-term recurrence relation $$T_{k+1}(x) = 2x T_k(x) - T_{k-1}(x) \,.$$
% \end{theorem}
% \begin{proof}{\parencite{atap}.}
%   For $k > 1$, we have
%   \begin{align*}
%     2x T_k(x) - T_{k-1}(x) & = 2x \cdot \frac{1}{2} (z^k + z^{-k}) - \frac{1}{2} (z^{k-1} + z^{-(k-1)})                     \\
%                            & = 2 \frac{1}{2}(z + z^{-1}) \cdot \frac{1}{2}(z^k + z^{-k}) - \frac{1}{2} (z^{k-1} + z^{-k+1}) \\
%                            & = \frac{1}{2} (z^{k+1} + z^{k-1} + z^{-k+1} + z^{-k-1}) - \frac{1}{2} (z^{k-1} + z^{-k+1})     \\
%                            & = \frac{1}{2} (z^{k+1} + z^{-(k+1)}) = T_{k+1}(x)
%   \end{align*}
% \end{proof}


The Jacobi polynomials are then defined from ${}_2F_1$ as follows:
\begin{definition}{Jacobi Polynomials}{jacobi-polynomials}
  Are given by \[J^{(a,b)}_n(x) = \mathrm{prefactor} \cdot {}_2F_1(...)\]
  So are defined using the {[}{[}Gaussian Hypergeometric Function{]}{]}.

  \hypertarget{nice-spectral-properties}{%
    \subsection{Nice Spectral Properties}\label{nice-spectral-properties}}

  \begin{itemize}
    \item
          Differentiation
    \item
          Three-Term Recurrence
    \item
          {[} {]} why are they better than just Chebyshev?
  \end{itemize}

  {[}{[}Gegenbauer Polynomials{]}{]} are a special case. And
  {[}{[}Chebyshev Polynomials{]}{]} are a special case of them.
\end{definition}

% \begin{definition}{Gegenbauer Polynomials}{gegenbauer-polynomials}
  \hypertarget{alias-ultraspherical-polynomials}
  {alias: Ultraspherical Polynomials}

  Are a special case of the Jacobi Polynomials (cf. \Cref{def:jacobi-polynomials}) and form an
  Orthonormal Basis (cf. \Cref{def:orthonormal-basis}) under the weight given by
  \[w(x)=(1+x)^\alpha\]
\end{definition}

% \begin{definition}{Chebyshev Polynomials}{chebyshev-polynomials}
  Of the first kind: \[T_k(x)\] Of the second kind: \[U_k(x)\] Also have a
  {[}{[}Three-Term Recurrence Relationship{]}{]}.
\end{definition}

\begin{definition}{Jacobi Matrix}{jacobi-matrix}
  aliases: Jacobi Operator

  The \href{https://en.wikipedia.org/wiki/Jacobi_operator}{Jacobi
    operator} is the matrix \(X \in \R^{N \times N}\) satisfying
  $$x \cdot P(x) = P(x) \cdot X^T$$
\end{definition}


% \pagebreak
\section{Working Towards a Solution}
Finally, now that we have established the basis functions, we can write down an ansatz $\rho: B_1(\vec{0}) \mapsto \R$ for the solution of \hyperref[def:the-problem]{the problem}, of the form
\begin{equation}
  \rho[\vec{\rho}](\vec{x}) = \rho(\vec{x}) := \left(1-\norm{\vec{x}}^{2}\right)^{m - \frac{\alpha + d}{2}} \sum_{k=0}^{N-1} \rho_k P_{k}^{\left(m - \frac{\alpha + d}{2},\frac{d-2}{2}\right)}(2 \norm{\vec{x}}^{2}-1)\,.
  \label{eq:ansatz}
\end{equation}
with $P_k^{(a, b)}$ the Jacobi polynomials and $\{\rho_k\}_{k=0, ..., N-1}$ the coefficients.
% TODO: is it alpha or beta in the exponent of (1-y\^{}2)?

The spectral method can then be written as a linear system of the coefficients $\vec{\rho}$ as we will see in the next section.
In order to establish said linear system, we first need to introduce the \textit{inverse fractional Laplacian}, helping us with the evaluation of the power law potential integral involving radial Jacobi polynomials given in \Cref{thm:theorem216}, the most important result of this chapter.

Let $(-\Delta)^{-\gamma}$ denote the inverse fractional Laplacian $\Delta := \nabla^2$ with power $\gamma \in (0, 1)$.
There are numerous equivalent definitions available (cf. \cite{2015-fractional-laplacian-definitions}), within the context of potential theory the Riesz potential definition (\Cref{def:riesz-potential}) is the most common.

% TODO: can we accept any gamma?
\begin{definition}{Riesz Potential}{riesz-potential}
  For a given function $u: \R^d \to \R$ and $\gamma \in \R$, its \textit{Riesz potential} $I_{\gamma}[u]$ is given by
  $$I_{\gamma}[u](\vec{x}) := \frac{2^{-\gamma} \Gamma(\tfrac{d-\gamma}{2})}{\pi^{d/2} \Gamma(\gamma/2)} \int_{\R^d} \frac{u(\vec{z})}{\norm{\vec{x}-\vec{z}}^{d-\gamma}} \dd\vec{z}\,.$$
\end{definition}

For $\gamma \in (0, d)$, the Riesz potential is equivalent to the inverse fractional Laplacian, so $(-\Delta)^{-\gamma} = I_\gamma$.
So in the case of positive power law kernels, the equivalence to the inverse fractional Laplacian does not apply.
Results on the integral will hold nevertheless and we move on to stating \Cref{thm:theorem216} from \cite{2021-arbitrary-dimensions} verbatim.

\begin{theorem}{Integration Theorem that needs a name}{theorem216}
  On the $d$-dimensional unit ball $B_1$ the power law potential, with power $\alpha \in(-d,2+2m-d)$, $m\in\mathbb{N}_0$ and $\beta>-d$, of the $n$-th weighted radial Jacobi polynomial $$(1-|y|^2)^{m-\frac{\alpha+d}{2}}P_n^{\left(m-\frac{\alpha+d}{2},\frac{d-2}{2}\right)}(2|y|^2-1)$$ reduces to a Gaussian hypergeometric function as follows:
  \begin{align*}
    \int_{B_1} & |x-y|^\beta (1-|y|^2)^{m-\frac{\alpha+d}{2}} P_n^{\left(m-\frac{\alpha+d}{2},\frac{d-2}{2}\right)}(2|y|^2-1) \mathrm{d}y                                                                                                                                                                                                                                                                              \\
               & = \tfrac{\pi ^{d/2} \Gamma \left(1+\frac{\beta}{2}\right) \Gamma \left(\frac{\beta+d}{2}\right) \Gamma \left(m+n-\frac{\alpha+d}{2}+1\right)}{\Gamma \left(\frac{d}{2}\right) \Gamma (n+1) \Gamma \left(\frac{\beta}{2}-n+1\right) \Gamma \left(\frac{\beta-\alpha}{2}+m+n+1\right)}{}_2F_1\left(\begin{matrix}n-\frac{\beta}{2}, -m-n+\frac{\alpha-\beta}{2} \\\frac{d}{2}\end{matrix};|x|^2\right).
  \end{align*}
\end{theorem}

\begin{proof}
  Using
  $$(-\Delta)^{-s} \left((R^2 - \norm{\vec{y}}^2)^q\right) = C_{q, s, d} R^{2q+2s} {}_2F_1\left(\begin{matrix}d/2 - s, -q-s \\ d/2 \end{matrix}; \frac{\norm{\vec{y}}^2}{R^2}\right)$$
  for $\norm{\vec{y}} \le R$ with $R = 1$ where $(-\delta)^{-s}$ denotes the inverse fractional Laplacian $\Delta := \nabla^2$ with power $s \in (0, 1)$ and $q \in \R^+$
  from \cite{2014-barenblatt} based on the \textit{Weber-Schafheitlin} integral of two Bessel functions given in \cite{1945-bessel-integral}.
  $$C_{q,s,d} = \frac{2^{-2s} \Gamma(q+1) \Gamma(d/2-s)}{\Gamma(d/2) \Gamma(q+s+1)}$$
  The Weber-Schafheitlin integrals are related to the fractional Laplacians of aforementioned functions because the Fourier transform of ${}_2F_1$ is a bessel function.

  The fractional Laplacian $(-\delta)^{-s}$ is defined by \cite{2015-fractional-laplacian-definitions}.
\end{proof}


% \Cref{thm:theorem216} gives an explicit expression for the main integral \(\mathcal{Q}^{\beta}: L \mapsto L\), an operator from the function space \(L\) to the function space \(L\), we are interested in:
% which is used to construct the spectral method operator \(Q^\beta\) (cf. \Cref{def:power-law-operator}), acting on the coefficients \(\vec{\rho}\).

Adding to our collection of tools, in order to solve the problem given in \Cref{def:the-problem} we need to normalise the solution by its mass.
The normalisation constant is given in \Cref{lemma:mass}, based only on a single coefficient $\rho_0$, allowing for a highly efficient renormalisation.
\begin{lemma}{Mass}{mass}
  For a given solution $\rho: B_1(\vec{0}) \mapsto \R$, its \textit{mass} $M \in \R$ is given by \Cref{eq:measure-mass}.
  Provided the appropriate ansatz given in \Cref{eq:ansatz}, an expansion of weighted radial Jacobi polynomials with coefficients $\rho_k$, has its \textit{mass} specifically given by
  \begin{align*}
    M := \int_{\supp(\rho)} \rho(y) \,\ddy = \frac{\pi^{\frac{d}{2}}\Gamma (a+1)}{\Gamma \left(a+\frac{d}{2}+1\right)} \rho_0\,,
  \end{align*}
  so solely depending on the first coefficient $\rho_0$.
\end{lemma}
\begin{proof}[Proof (from \cite{2021-arbitrary-dimensions})]
  The domain and radial symmetry of this problem suggests the use of hyperspherical coordinates:
  \begin{align*}
    M_1 = \int_{B_1(\vec{0})} \rho(y) dy & = \sum_{n=0}^{N-1} \rho_{n} \int_{B_1(\vec{0})} (1-|y|^2)^a P_n^{(a,\frac{d-2}{2})}(2|y|^2-1) dy                                            \\
                                         & =\sum_{n=0}^{N-1} \rho_{n} \int_{\partial B_1(\vec{0})} \int_{r=0}^1 (1-r^2)^a P_n^{(a,\frac{d-2}{2})}(2r^2-1) r^{d-1} dr d\sigma(\omega)   \\
                                         & = \frac{2\pi^\frac{d}{2}}{\Gamma(\frac{d}{2})} \sum_{n=0}^{N-1} \rho_{n} \int_{r=0}^1 (1-r^2)^a P_n^{(a,\frac{d-2}{2})}(2r^2-1) r^{d-1} dr,
  \end{align*}
  where ${2\pi^\frac{d}{2}} / {\Gamma(\frac{d}{2})}$ is the surface area of the $d-1$ dimensional sphere (cf. \Cref{lemma:surface-area}).
  % Note that working with a finite expansion ($N < \infty$), we automatically have
  % $$\int_{B_1} \sum_{k=0}^{N-1} |\rho_k (1-|y|^2)^a P_k^{(a,\frac{d-2}{2})}(2|y|^2-1)| \,\ddy < \infty\,,$$
  % so the exchange of integration and infinite sum in the first line is justified by the Fubini-Tonelli theorem.

  The integral in the resulting expression can be evaluated easily by reversing the quadratic shift of the Jacobi polynomials via transformation $2r^2-1 = t$:
  \begin{align*}
    \sum_{n=0}^{N-1} \rho_{n} \int_0^1 (1-r^2)^a & P_n^{(a,\frac{d-2}{2})}(2r^2-1) r^{d-1} dr                                                                                \\&= \frac{1}{4} \sum_{n=0}^{N-1} \rho_{n} \int_{-1}^1 \frac{(1-t)^a}{2^a}\frac{(1+t)^{\frac{d-2}{2}}}{2^{\frac{d-2}{2}}} P_n^{(a,\frac{d-2}{2})}(t) dt\\
                                                 & = 2^{\frac{-2-2a-d}{2}} \sum_{n=0}^{N-1} \rho_{n} \int_{-1}^1 (1-t)^a (1+t)^{\frac{d-2}{2}} P_n^{(a,\frac{d-2}{2})}(t) dt \\
                                                 & = \frac{\Gamma (a+1) \Gamma \left(\frac{d}{2}\right)}{2 \Gamma \left(a+\frac{d}{2}+1\right)} \rho_0,
  \end{align*}
  where the last equality relies on the classical orthogonality condition of the Jacobi polynomials in \Cref{eq:jacobi-orthogonality-condition} where we remind ourselves that $P_0(t) = 1$. Combining this with the prior expression yields the stated result.
\end{proof}

\begin{lemma}{Surface area of the hypersphere}{surface-area}
  The surface area of $\partial B_R(\vec{0})$ is given by
  $$A_{d-1}(R) = \frac{\dd}{\dd R} V_d(R) = \frac{2\pi^\frac{d}{2}}{\Gamma(\frac{d}{2})} R^{d-1}\,.$$
\end{lemma}
\begin{proof}
  \hierKoennteIhreWerbungStehen
\end{proof}


\pagebreak
\section{Derivation of the Operator}
Similar to \Cref{def:power-law-potential}, we can define the (single) power law operator $\mathcal{Q}^\beta$:

% TODO: explain why we are not solving \int\int but only the following
\begin{definition}{Power Law Operator $\mathcal{Q}^\beta$}{power-law-operator}
  The power law operator $\mathcal{Q}^\beta: \functionspace \mapsto \functionspace$ is given by
  $$\mathcal{Q}^\beta[\rho](\vec{x}) := \int \norm{\vec{x}-\vec{y}}^\beta \,\dd\rho(\vec{y}) = \int_{\supp(\rho)} \norm{\vec{x}-\vec{y}}^\beta \rho(\vec{y}) \,\dd\vec{y}\,,$$
\end{definition}
% Either the attractive or the repulsive operator can be sparse.
% Obtained using \Cref{thm:theorem216}.

The operator $\mathcal{Q}^\beta$ acting on an equilibrium measure $\rho(\vec{x})$ returns the energy $\tilde{E}(\vec{x}) = \mathcal{Q}^\beta[\rho](\vec{x})$ at a point $\vec{x} \in B_1(\vec{0})$ in our normalised domain.

We start by only considering a single power law operator (out of two in the case of an attractive-repulsive interaction potential $K_{\alpha, \beta}$).
Substituting our ansatz given in \Cref{eq:ansatz} into $\mathcal{Q}^\beta[\rho]$, we obtain
\begin{equation}
  \mathcal{Q}^{\beta}[\rho](x) = \sum_{k=0}^{N-1} \rho_{k} \int_{B_1(\vec{0})} \norm{\vec{x}-\vec{y}}^{\beta} \weight{y}\jacobi{y} \,\dd\vec{y}\,,
  \label{eq:theorem216-in-derivation}
\end{equation}
luckily containing the integral evaluated in \Cref{thm:theorem216}.

We are now interested in a numerical representation of the operator $\mathcal{Q}^\beta$ acting on the function $\rho \in \functionspace$, so an equivalent (linear) operator $Q^\beta: \R^N \mapsto \R^N$ acting on the coefficients $\rho_k \in \R,\, k = 0, ..., N-1$.
As every finite-dimensional linear operator must have a matrix representation, we look for a $Q^\beta \in \R^{N \times N}$ such that
$$\mathcal{Q}^\beta[\rho](\vec{x}) = \jacobivec{x} \cdot Q^\beta \vec{\rho}\,,$$
where $\jacobivec{x} \in \R^N$ is the vector of radial Jacobi polynomials $P^{(a, b)}_0(x)$, $P^{(a, b)}_1(x)$, ..., $P^{(a, b)}_{N-1}(x)$ evaluated at $2\norm{\vec{x}}^2 - 1$ as introduced in and after \Cref{def:jacobi-polynomials}.
Note that in the context of linear combinations of Jacobi polynomials, we will use zero-based indexing for vectors and matrices due to the convention that the first orthogonal polynomial is usually denoted by $p_0(x) = 1$, in line with $\deg(p_k) = k$.

Based on the three-term recurrence relationship (cf. \Cref{thm:three-term-recurrence-relationship}), one can even determine an explicit relationship between the coefficients in the Jacobi expansion by considering the Jacobi matrix (cf. \Cref{remark:jacobi-matrix}).
We use this recurrence relationship in our implementation to significantly speed up the construction of the operator (cf. \Cref{sec:runtime-analysis}).
The recurrence coefficients used are due to \cite{2021-arbitrary-dimensions} and \cite{2023-olver-equilibrium-measures-jl}.

% TODO: this section has weird indexing (start at 0 or 1?) and needs some work.
Therefore, starting from \Cref{eq:theorem216-in-derivation}, we obtain
\begin{align*}
  \mathcal{Q}^\beta[\rho](\vec{x}) & = \sum_{k=0}^{N-1} \rho_k \mathcal{Q}^\beta[wP_k](\vec{x}) = \sum_{k=0}^{N-1} \rho_k \sum_{j=0}^{N-1} q^{\beta}_{kj} \jacobi{x} \\
                                   & = \sum_{j=0}^{N-1} \sum_{k=0}^{N-1} \rho_k q^{\beta}_{kj} \jacobi{x}\,,
\end{align*}
which we will rewrite in matrix-form,
\begin{align*}
  \mathcal{Q}^\beta[\rho](\vec{x}) & = \vec{P}(\vec{x}) \cdot
  \begin{pmatrix}
    \sum_{k=0}^{N-1} \rho_k q^{\beta}_{k,1} \\
    \vdots                                  \\
    \sum_{k=0}^{N-1} \rho_k q^{\beta}_{k,N}
  \end{pmatrix} = \vec{P}(\vec{x}) \cdot
  \underbrace{
    \begin{pmatrix}
      q^{\beta}_{00}    & \dots  & q^{\beta}_{0,N-1}   \\
      \vdots            & \ddots & \vdots              \\
      q^{\beta}_{N-1,0} & \dots  & q^{\beta}_{N-1,N-1} \\
    \end{pmatrix}
  }_{=: Q^\beta}
  \begin{pmatrix}
    \rho_0 \\
    \vdots \\
    \rho_{N-1}
  \end{pmatrix}                                                              \\
                                   & = \jacobivec{x} \cdot Q^\beta \vec{\rho}
\end{align*}
where we use $\vec{P}(\vec{x}) = \jacobivec{x}$ as a shorthand giving us the form of the operator matrix.
Its first row, the set of coefficients for the constant Jacobi polynomial $P_0^{(a,b)}$ of each expansion of $Q^\beta$, weighted by the solution coefficients, adds up to the total energy $E$ (coefficient of the constant polynomial).
All remaining rows of $Q^\beta$ contain coefficients for polynomials of degree at least 1, so their weighted contributions must add up to 0 in order for the total energy $\tilde{E}(\vec{x})$ to remain constant.

For the attractive-repulsive interaction potential $K_{\alpha,\beta}$, because each column consists of an expansion of the Jacobi polynomials, we have $q^{\beta}_{0k} = I_{m,k}^{\alpha,\beta}$.

\begin{figure}[H]
  \centering
  \includegraphics[width=\linewidth]{results/attrep/attractive-repulsive-operators.pdf}
  \caption[Attractive and repulsive operators.]{The attractive and repulsive operators (matrices) as given in \Cref{def:power-law-operator}, the matrix values are shown in a $\log_{10}$ colour scale. Due to the choice of basis, the attractive operator is exactly banded. The repulsive parameter is only approximately banded, which the spy plots effectively demonstrate.}
  \label{fig:attractive-repulsive}
\end{figure}

The bandedness of the attractive operator in \Cref{fig:attractive-repulsive} is due to the three-term recurrence relationship of the Jacobi polynomial basis (cf. \Cref{thm:three-term-recurrence-relationship} and \Cref{def:jacobi-polynomials}).
% TODO.. explain

For the attractive-repulsive interaction potential $K_{\alpha,\beta}(r)$, the full operator is given by
\begin{align*}
  \mathcal{Q}_{\alpha, \beta}[\hat{\rho}](\hatvec{x}) & := \int_{B_R(\vec{0})} K_{\alpha,\beta}\left(\norm{\hatvec{x} - \hatvec{y}}\right) \hat{\rho}(\hatvec{y}) \,\dd\hatvec{y}                                                               \\
                                                      & = \int_{B_R(\vec{0})} \left(\frac{\norm{\hatvec{x} - \hatvec{y}}^\alpha}{\alpha} - \frac{\norm{\hatvec{x} - \hatvec{y}}^\beta}{\beta}\right) \hat{\rho}(\hatvec{y}) \,\dd\hatvec{y}     \\
                                                      & = \int_{B_1(\vec{0})} \left(\frac{R^\alpha \norm{\vec{x} - \vec{y}}^\alpha}{\alpha} - \frac{R^\beta \norm{\vec{x} - \vec{y}}^\beta}{\beta}\right) \hat{\rho}(R\vec{y}) \, R^d\dd\vec{y} \\
                                                      & = R^d\int_{B_1(\vec{0})} \left(\frac{R^\alpha}{\alpha}\norm{\vec{x} - \vec{y}}^\alpha - \frac{R^\beta}{\beta} \norm{\vec{x} - \vec{y}}^\beta\right) \rho(\vec{y}) \,\dd\vec{y}          \\
                                                      & = \frac{R^{\alpha+d}}{\alpha} \mathcal{Q}^\alpha[\rho](\vec{x}) - \frac{R^{\beta+d}}{\beta} \mathcal{Q}^\beta[\rho](\vec{x})\,,
\end{align*}
where one needs to carefully handle the variable transform with $\dd\hatvec{y} = R^d \dd\vec{y}$ in $d$ dimensions whereas the vectors themselves obey $\hatvec{y} = R \vec{y}$ as established previously.
In matrix form that is,
\begin{equation}
  Q_{\alpha, \beta} := \frac{R^{\alpha+d}}{\alpha} Q^\alpha - \frac{R^{\beta+d}}{\beta} Q^\beta\,,
  \label{eq:full-attrep-operator}
\end{equation}
for some interval radius $R \in \R^+$.
The full operator for a set of example parameters is depicted in \Cref{fig:attrep-operator}.
As one can see, it is approximately banded and therefore, sparse (note that the colouring is done on a log-scale).

\begin{figure}[H]
  \centering
  \includegraphics[width=0.5\linewidth]{results/attrep/full-operator.pdf}
  \caption[Combination of the attractive-repulsive operators]{Spy plot of $Q_{\alpha, \beta}$, the combination of the attractive-repulsive operators given in \Cref{fig:attractive-repulsive}. Inverting this operator and applying it to $(1, 0, ..., 0)^T \in \R^N$ will yield the un-normalised coefficients $\rho_k$ of the solution expansion given in \Cref{eq:ansatz}.}
  \label{fig:attrep-operator}
\end{figure}


\pagebreak
% Full Section:
\section{Solving a Linear System}
Once the operator is computed, we are now looking for a set of solution coefficients $\vec{\rho} \in \R^N$ such that the total energy $E = E_{\rm kin} + U = U = U_K[\hat{\rho}]$ (in the presence of friction, the kinetic energy will eventually dissipate, cf. \Cref{chap:particle-interaction-theory}) on the domain $D = B_R(\vec{0})$ is constant.
That means, we are looking for $\vec{\rho} \in \R^N$ such that
\begin{equation}
  \mathcal{Q}[\rho](\vec{x}) = \tilde{E}(\vec{x}) = E\,,
\end{equation}
where we can expand $\tilde{E}(\vec{x}) = \jacobivec{x} \cdot \vec{E}$ into Jacobi polynomials with coefficients $\vec{E} = E \vec{e}_1 = (E, 0, ..., 0)^T$ such that the energy is constant along the entire domain, so $\tilde{E}(\vec{x}) = E \cdot P_0^{(a, b)}\left(\jacobiarg{x}\right) = E$.
In matrix form, that is
$$Q \vec{\rho} = \vec{E} \qLRq \begin{pmatrix}
    q_{00}    & \dots  & q_{0,N-1}   \\
    \vdots    & \ddots & \vdots      \\
    q_{N-1,0} & \dots  & q_{N-1,N-1} \\
  \end{pmatrix} \begin{pmatrix}
    \rho_0 \\
    \vdots \\
    \rho_{N-1}
  \end{pmatrix} = \begin{pmatrix}
    E       \\
    \vec{0} \\
    0
  \end{pmatrix}\,.$$

This equation $Q \vec{\rho} = \vec{E}$ contains two unknowns, so we need a second equation to find the full solution $\rho \in \functionspace$ and, thereby, $\hat{\rho} \in \functionspacehat$.
The second piece of information we are looking for is the mass given in \Cref{eq:measure-mass}, which is set to $M = 1$. We start by dividing $Q \vec{\rho} = E \vec{e}_1$ by the unknown energy $E$,
\begin{equation}
  Q \frac{\vec{\rho}}{E} = Q \vec{\tilde{\rho}} = \vec{e}_1\,,
  \label{eq:the-linear-system}
\end{equation}
which we can efficiently solve using readily available linear system solvers.
After solving, we ensure $M\left[\rho[\vec{\tilde{\rho}}]\right] = 1$, using \Cref{lemma:mass}, leading us to our final equilibrium distribution $\rho \in \functionspace$.

% The linear system is solved using a QR-decomposition.
% TODO: what would Yuji say?

Hence the total potential (energy) of a given solution $\vec{\rho}$ is obtained by
\begin{equation}
  E(R) = \{Q_{\alpha,\beta} \vec{\rho}\}_1 = \sum_{k=0}^{N-1} \rho_k \left(\frac{R^{\alpha+d}}{\alpha} \bar{I}_{m,k}^{\alpha,\alpha} - \frac{R^{\beta+d}}{\beta} \bar{I}_{m,k}^{\alpha,\beta}\right)\,,
  \label{eq:total-energy-for-ansatz}
\end{equation}
with $\bar{I}_{m,n}^{\alpha,\alpha}$ the 0th coefficient of a Jacobi polynomial expansion of $I_{m,n}^{\alpha,\alpha}(\vec{x})$ (cf. \Cref{thm:theorem216}) as given in \Cref{eq:0th-coeff-of-I}.

\subsection{Tikhonov Regularisation}
For larger system sizes $N \gg 1$, numerical instability might become a concern because Fredholm equations of the first kind posed on Banach spaces are Hilbert-Schmidt and therefore compact, so in the infinite case they cannot be inverted \parencite{2021-arbitrary-dimensions}.
One way to address this is by regularisation of the system, which turns the first-kind into a second-kind equation.

Instead of solving the original linear system $Q \vec{\tilde{\rho}} = \vec{e}_1$, one can solve the normal system, a transformation from the original coordinates to a modified coordinate system.
Left-multiplying the matrix normal of the operator, it is given by
$$Q^* Q \vec{\tilde{\rho}} = Q^* \vec{e}_1\,.$$

The simplest possible Tikhonov regularisation, also referred to as Ridge regression \parencite{1970-ridge-regression}, can be achieved by perturbing the normal equation by a small $s \in \R^+$, $s \ll 1$.
The modified system therefore becomes
$$(Q^* Q + sI) \vec{\tilde{\rho}} = Q^* \vec{e}_1\,,$$
with $I \in \R^{N \times N}$ the identity matrix.
Solving it instead of the original system given in \Cref{eq:the-linear-system} results in a slight error in the solution, depending on the value of $s$, but the Tikhonov regularisation significantly improves the coefficient decay when adding more terms to the expansion.
This coefficient decay is expected, after some $N$, further added Jacobi polynomials in the expansion should not make a significant difference, hence their coefficients should shrink for large $N$.
The behaviour of our solver with and without regularisation is shown in \Cref{fig:convergence-to-analytic,fig:coefficients}.


\section{Results}
\begin{figure}[H]
  \centering
  \includegraphics[width=0.8\linewidth]{results/attrep/solution-increasing-order.pdf}
  \caption[Solutions of increasing orders]{Particle density distribution function solutions $\rho$ of increasing order $N$ to the attractive-repulsive problem with interaction potential $K_{alpha, \beta}(r)$, $\alpha = 2.5$ and $\beta = 1.2$. Reflected along the y-axis for better visibility of the domain.}
  \label{fig:solution-increasing-order}
\end{figure}

% Full Section:
\section{Outer Optimisation Routine}
The unconstrained outer optimisation over the scalar value $R \in \R^+$, the radius of our domain $B_R(\vec{0})$, is carried out using \href{https://github.com/JuliaNLSolvers/Optim.jl}{Optim.jl}'s implementation \parencite{2023-optim-jl} of the \gls{lbfgs} optimisation method \parencite{1989-lbfgs}, an extension of BFGS for low-memory usage, using an estimate for the gradient based on \gls{ad} techniques.

As part of a comparison between multiple optimisation approaches, \gls{lbfgs} outperformed the Nelder-Mead and Newton trust region methods for our case, converging extremely quickly within only 3 iterations, 10 function calls and 10 evaluations of the gradient.
While the downhill simplex method by \citeauthor{1965-nelder-mead} did not converge to the desired local minimum (cf. \Cref{fig:outer-optimisation}), the trust region method using Newton's method to solve a quadratic model for each subproblem \parencite{1982-trust-region} also converged in only 3 iterations with only 4 function, gradient and Hessian evaluations.
Again, the values of the gradient and Hessian (in the case of a one-dimensional optimisation, simply the first and second derivatives), are obtained using \glstext{ad}.

The \gls{lbfgs} method converged with $\norm{\nabla E(R)} \approx 10^{-11}$ while the Newton trust region method converged at $\norm{\nabla E(R)} \approx 10^{-9}$.
The entire optimisation routine with \gls{lbfgs}, including function and gradient evaluations (solving a $12 \times 12$ linear system), takes \SI{28 \pm 4}{\milli\second} on an Intel\textregistered \, i7-5600U CPU running at \SI{2.6}{\giga\hertz}.

\begin{figure}[H]
  \centering
  \includegraphics[width=0.8\linewidth]{results/known-analytic/outer-optimisation.pdf}
  \caption[Outer Optimisation Routine]{The total potential $U$ as a function of the support radius $R$. This is the goal function minimised by the outer optimisation routine.}
  \label{fig:outer-optimisation}
\end{figure}

Note that using this setup, the operators themselves do not need to be recomputed upon a change in $R$, cf. \Cref{eq:full-attrep-operator}.
The provided implementation uses \gls{lru} caching to automatically store operators for a given parameter set and order $N$.


\pagebreak
% Full Section:
\section{Comparison with Analytic Solutions}
As introduced in \Cref{sec:analytical-solutions}, there are some analytical solutions available which allow us to perform further analysis of the numerical method in these special cases.

The major advantage of a spectral method is its so-called \textit{spectral convergence}, sometimes also referred to as exponential convergence, cf. \Cref{def:spectral-convergence}.
\begin{definition}{Spectral Convergence}{spectral-convergence}
  An \(N\)-point approximation \(\varphi_N\) of a function f converges to \(f\) at spectral speed if \(|\varphi_N -f|\) decays pointwise in \([-1, 1]\) faster than \(\mathcal{O}(N^{-p})\) for any \(p = 1, 2, . . .\) so \(p \in \mathbb{N}\).
\end{definition}


For a given set of parameters with a known analytic solution (cf. \Cref{sec:analytical-solutions}), we compare growing orders of the spectral method's solution with the analytic expression in a set of 200 points and plot the pointwise error, cf. \Cref{fig:analytic-solution}.
The figure also shows how the outer optimisation routine will approach the optimal $R_{\rm opt}$ closer and closer for growing orders $N$.

\begin{figure}[H]
  \centering
  \includegraphics[width=\linewidth]{results/known-analytic/analytic-solution.pdf}
  \caption[Comparison with analytical solutions and error]{
    The analytic solution $\rho(x)$ given in \Cref{eq:analytical-solution-alpha-equal-2} compared to the (spectral method) solutions of different order $N$.
    The ``arches'' occur as a result of the roots of $\rho(x) - \rho_N(x)$, their number approximately equals the order $N$ (a polynomial of degree $N$ has at most $N$ roots).
  }
  \label{fig:analytic-solution}
\end{figure}

The reason for this fast convergence is the choice of weighted basis in \Cref{eq:ansatz} compared to the form of the analytical solution in \Cref{eq:analytical-solution-alpha-equal-2}.
% TODO: when added theorem216() with limit-behaviour
There are more analytic solutions available for other parameter ranges, which we will not analyse within the scope of this dissertation.

\begin{figure}[H]
  \centering
  \includegraphics[width=0.8\linewidth]{results/known-analytic/convergence-to-analytic.pdf}
  \caption[Convergence to analytic solution]{Convergence of the numerical solution to the known analytic solution (cf. \Cref{eq:analytical-solution-alpha-equal-2}) in a special case where it is known, squared error plotted as a function of the highest order in the expansion $N$.}
  \label{fig:convergence-to-analytic}
\end{figure}


\section{Discussion}
\begin{figure}[H]
  \centering
  \includegraphics[width=0.8\linewidth]{results/attrep/convergence.pdf}
  \caption[Step-by-step convergence of solutions compared to order 24]{Step-by-step convergence of numerical solutions $\rho_N(x)$ as compared to $\rho_{24}(x)$, visualised using the squared error of the pointwise evaluation of both functions in $200$ points.}
  \label{fig:convergence}
\end{figure}

\begin{figure}[H]
  \centering
  \includegraphics[width=0.9\linewidth]{results/attrep/varying-parameters.pdf}
  \caption[Varying parameters in the solver]{
    Varying different parameters in the solver to demonstrate their effect.
    See also, \Cref{fig:varying-R-solutions}.
  }
  \label{fig:varying-parameters}
\end{figure}

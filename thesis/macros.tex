\makenoidxglossaries
\newacronym{fcc}{fcc}{face-centered cubic}
\newacronym{ml}{ML}{Machine Learning}
\newacronym{gd}{GD}{Gradient Descent}
\newacronym{gui}{GUI}{Graphical User Interface}
\newacronym{lru}{LRU}{Least Recently Used}
\newacronym{pde}{PDE}{Partial Differential Equation}
\newacronym{ad}{AD}{Automatic Differentiation}
\newacronym{lbfgs}{LBFGS}{Limited-memory Broyden-Fletcher-Goldfarb-Shanno}

\usepackage{xifthen}
\newcommand{\name}[1]{\textsc{#1}}
\newcommand{\inv}{^{-1}}
\newcommand{\qRq}{\quad\Rightarrow\quad}
\newcommand{\qLRq}{\quad\Leftrightarrow\quad}
\newcommand{\sfrac}[2]{{#1 / #2}}
\newcommand{\functionspace}{\mathcal{L}}
\newcommand{\functionspacehat}{\hat{\mathcal{L}}}
\renewcommand{\norm}[1]{\left\lVert#1\right\rVert_{\scriptscriptstyle 2}}
\newcommand{\jacobiarg}[1]{2 \norm{\vec{#1}}^2 - 1}
\newcommand{\jacobi}[2][]{P_{\ifthenelse{\isempty{#1}}{k}{#1}}^{(a, b)}\left(\jacobiarg{#2}\right)}
\newcommand{\jacobivec}[1]{\vec{P}^{(a, b)}\left(\jacobiarg{#1}\right)}
\newcommand{\weight}[2][]{\left(1-\norm{\vec{#2}}^2\right)^{\ifthenelse{\isempty{#1}}{a}{#1}}}
\newcommand{\goesto}{\rightarrow}
\newcommand{\hatvec}[1]{\hat{\vec{#1}}}

\newcommand{\hierKoennteIhreWerbungStehen}{
  \begin{center}
    \bfseries \textcolor{themecolor2}{To be done / included.}
  \end{center}
}
